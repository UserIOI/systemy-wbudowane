\documentclass[12pt]{article}
\usepackage{polski}
\usepackage[margin=2cm]{geometry}


\title{Systemy wbudowane\\ Projekt drona}
\author{Paweł Grzegorzewski, Jan Nawrat, Paweł Haraburda}
\date{}

\begin{document}
\maketitle

\section{Gdzie system jest wykorzystywany. (GDZIE)}
Korzystać z systemu można w obszarach zamkniętych jak i otwartych. Między innymi: obszary zurbanizowane, terenty wiejskie, obszary leśne oraz górskie. Urządzenie nie nadaje się do korzystania w wodzie. 

\subsection{Ograniczenia systemu.}
Korzystając z urządzenia trzeba brać pod uwagę czynniki takie jak: 
\begin{itemize}
    \item Pogoda $-$ przy dużym wietrze mogą wystąpić problemy ze sterownością, przy wzmożonym deszczu może dojść do zwarć w systemie, bądź w momencie burz do uderzenia piorunem. W sytuacji dużego zachmurzenia lub mgły obraz z kamery może być niewyraźny oraz jest możliwe utrata widoczności drona. Korzystając z urządzenia w niższuch temperaturach prawdopodobne jest szybsze wyczerpanie akumulatora. 
    \item Wysokość $-$ w momencie osiągania większych wysokości dron stanowi poważniejsze zagrożenie w momencie awarii systemu. Trzeba też brać pod uwagę możliwe kolizje z innymi statkami powietrznymi (innymi dronami, samolotami).
    \item Zasięg $-$ dron posiada ograniczony zasięg latania spowodowanym utratą sygnału z kontrolerem na dalszych odległościach.
    \item Prawne $-$ każde państwo posiada własne regulacje prawne dotyczące latania dronami oraz innymi bezzałogowymi statkami powietrznymi takie jak limit wysokości latania, brak możliwości latania w miastach bądź nad tłumami.
\end{itemize}

\section{Dla kogo system jest skierowany. (KTO)}
\begin{itemize}
    \item Serwisant $-$ naprawa urządzenia, wymiana części, testowanie działania systemu.
    \item Użytkownik $-$ rekreacyjne/ekstremalne latanie dronem, robienie zdjęć/filmów, kalibracja oraz ładowanie urządzenia, podgląd z kamery urządzenia na telefonie za pomocą dedykowanej aplikacji.  
\end{itemize}

\end{document}