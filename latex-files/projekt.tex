\documentclass[12pt]{article}
\usepackage{polski}
\usepackage[margin=2cm]{geometry}


\title{Systemy wbudowane\\ Projekt drona}
\author{Paweł Grzegorzewski, Jan Nawrat, Paweł Haraburda}
\date{}

\begin{document}
\maketitle
\section{Co?}
System zajmuje się obsługą Bezzałogowego statku powietrznego (w skócie - BSP, pot. dron) **tutaj mozna wrzucic to do slownika**, odpowiada za umożliwienie lotu oraz sterowania zewnętrznego. Osoba kontrolująca obiekt będzie mieć możliwość podglądu do widoku z pokładu na bieżąco oraz wykonywania fotografii. BPS powinnien zapisywać lokalizacje miejsca startowego i w wypadku ewentualnej awarii lub stracenia połaczenia z modułem sterującym powracać do zapamietanej lokalizacji. Kontroler, będzie dawał możliwość poruszania sie w 3 ośiach, zdalnej obserwacji oraz wykonywania fotografii z pokładu. Osoba niewykwalifkiowana powinna mieć możliwość wykonanai kalibracji.
\section{Jak?}
\begin{enumerate}
    \item Łączność modułu sterującego z BSP: wyposażenie drona w moduł Wi-Fi, łączenie się poprzez aplikacje mobilną i podłączenie urządzenia mobilnego do dedykowanego kontrolera
    \item Sterowanie Lotem: Funkcja sterowania lotem jest realizowana poprzez system stabilizacji lotu, który wykorzystuje algorytmy kontroli lotu i czujniki inercyjne, zapewniając płynne i precyzyjne manewry BSP.
    \item Wizualizacja na Żywo: System umożliwia wizualizację na żywo poprzez transmisję obrazu z kamery zainstalowanej na pokładzie drona do interfejsu użytkownika w aplikacji.
    \item Wykonywanie Fotografii: Funkcja wykonywania fotografii jest realizowana poprzez sterowanie zintegrowanym aparatem lub kamerą na pokładzie drona za pomocą odpowiednich poleceń wysyłanych z kontrolera użytkownika, dwa przyciski na kontrolerze z lampka kontrolną.
    \item Zapisywanie Lokalizacji Startowej: BSP zapisuje lokalizację miejsca startowego poprzez wykorzystanie systemów GPS oraz pamięci wewnętrznej, co pozwala na szybkie odnalezienie punktu startowego w przypadku konieczności powrotu.
    \item Automatyczne Powracanie do Lokalizacji Startowej: Funkcja automatycznego powracania do lokalizacji startowej jest realizowana poprzez algorytmy nawigacyjne, które wykorzystują zapisaną lokalizację startową i sterują BSP w kierunku tego punktu w przypadku utraty połączenia z kontrolerem.
    \item Sterowanie: Kontroler BSP umożliwia sterowanie w trzech osiach poprzez odpowiednie manipulowanie dwoma drążkami (jeden w osiach x i z, drugi w osi y), które przekazują sygnały sterujące do odpowiednich silników i powierzchni sterowych drona.
    \item Kalibracja dla Osób Niewykwalifikowanych: intuicyjna procedura kalibracji z instrukcją krok po kroku. Diody kontrolne na kazdym ramieniu z silnikiem, pomagające w kalibracji oraz zapewniające lepszą widoczność BSP.
\end{enumerate}
\section{Gdzie system jest wykorzystywany. (GDZIE)}
Korzystać z systemu można w obszarach zamkniętych jak i otwartych. Między innymi: obszary zurbanizowane, terenty wiejskie, obszary leśne oraz górskie. Urządzenie nie nadaje się do korzystania w wodzie. 

\subsection{Ograniczenia systemu.}
Korzystając z urządzenia trzeba brać pod uwagę czynniki takie jak: 
\begin{itemize}
    \item Pogoda $-$ przy dużym wietrze mogą wystąpić problemy ze sterownością, przy wzmożonym deszczu może dojść do zwarć w systemie, bądź w momencie burz do uderzenia piorunem. W sytuacji dużego zachmurzenia lub mgły obraz z kamery może być niewyraźny oraz jest możliwe utrata widoczności drona. Korzystając z urządzenia w niższuch temperaturach prawdopodobne jest szybsze wyczerpanie akumulatora. 
    \item Wysokość $-$ w momencie osiągania większych wysokości dron stanowi poważniejsze zagrożenie w momencie awarii systemu. Trzeba też brać pod uwagę możliwe kolizje z innymi statkami powietrznymi (innymi dronami, samolotami).
    \item Zasięg $-$ dron posiada ograniczony zasięg latania spowodowanym utratą sygnału z kontrolerem na dalszych odległościach.
    \item Prawne $-$ każde państwo posiada własne regulacje prawne dotyczące latania dronami oraz innymi bezzałogowymi statkami powietrznymi takie jak limit wysokości latania, brak możliwości latania w miastach bądź nad tłumami.
\end{itemize}

\section{Dla kogo system jest skierowany. (KTO)}
\begin{itemize}
    \item Serwisant $-$ naprawa urządzenia, wymiana części, testowanie działania systemu.
    \item Użytkownik $-$ rekreacyjne/ekstremalne latanie dronem, robienie zdjęć/filmów, kalibracja oraz ładowanie urządzenia, podgląd z kamery urządzenia na telefonie za pomocą dedykowanej aplikacji.  
\end{itemize}

\end{document}
