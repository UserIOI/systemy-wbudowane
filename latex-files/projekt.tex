\documentclass[12pt]{article}
\usepackage{polski}
\usepackage[margin=2cm]{geometry}


\title{Systemy wbudowane\\ Projekt drona}
\author{Paweł Grzegorzewski, Paweł Haraburda,  Jan Nawrat}
\date{}

\begin{document}
\maketitle

\section{Słownik pojęć}

W dokumentacji używane będą następujące pojęcia:

\begin{itemize}
    \item BSP $ - $ bezzałogowy statek powietrzny (ang. \textit{unmanned aerial vehicle, skr. UAV}), statek powietrzny bez możliwości zabierania pasażerów, w tym przypadku pilotowany zdalnie
    \item dron $ - $ inaczej BSP
    \item kontroler $ - $ niewielkie urządzenie umożliwiające sterowanie BSP na odległość poprzez RC, używające urządzenia z systemem Android lub iOS jako wyświetlacza
    \item RC $ - $ Radio Control, zdalne sterowanie realizowane drogą radiową
\end{itemize}

\section{Jakie są założenia projektu (CO)}
System zajmuje się obsługą BSP z kamerą na pokładzie, odpowiada za umożliwienie lotu oraz sterowania zewnętrzengo. Sterowanie dronem będzie odbywało się z użyciem kontrolera. Użytkownik będzie miał możliwość sterowania lotem w trzech osiach oraz zapisywania fotografii. Opcjonalnie do kontrolera będzie można podłączyć urządzenie mobilne z systemem Android lub iOS i uzyskać dostęp do poglądu z kamery pokładowej na żywo. W przypadku awarii lub utraty połączenia z kontrolerem dron podejmie próbę powrotu do miejsca startu. Wstępna kalibracja BSP będzie możliwa do wykonania przez użytkownika bez kwalifikacji ani wcześniejszego doświadczenia.

\section{W jaki sposób założenia zostaną zrealizowane (JAK)}
\begin{enumerate}
    \item Łączność modułu sterującego z BSP $ - $ dron zostanie wyposażony w moduł RC, za pomocą którego będzie łączył się z kontrolerem. Poprzez użycie połączenia USB z kontrolerem i dedykowanej aplikacji obraz z kamery na pokładzie będzie mógł być odbierany i wyświetlany na urządzeniu mobilnym.
    \item Sterowanie $ - $ kontroler będzie umożliwiał sterowanie BSP w trzech osiach poprzez odpowiednie manipulowanie dwoma drążkami (jeden w osiach x i z, drugi w osi y). Ruchy te będą odpowiednio interpretowane poprzez oprogramowanie na pokładzie drona i wysyłane będą sygnały sterujące do odpowiednich silników i powierzchni sterowych drona.
    \item Wspomaganie lotu $ - $ dron będzie wyposażony w system stabilizacji lotu, który wykorzystuje algorytmy kontroli lotu i czujniki inercyjne, zapewniając płynne i precyzyjne manewry.
    \item Podgląd na żywo $ - $ system będzie umożliwiał transmisję obrazu z kamery zainstalowanej na pokładzie drona do dedykowanej aplikacji  w czasie rzeczywistym.
    \item Wykonywanie fotografii $ - $ możliwe będzie wykonanie fotografii zintegrowaną kamerą na pokładzie drona. Kontroler będzie wyposażony w dwa przyciski oraz lampkę kontrolną przeznaczone do obsługi tej funkcji.
    \item Zapisywanie lokalizacji startowej $ - $ BSP będzie zapisywał lokalizację miejsca startowego w pamięci wewnętrznej poprzez wykorzystanie systeu GPS, co pozwali na szybkie odnalezienie punktu startowego w przypadku konieczności powrotu.
    \item Automatyczne powracanie do lokalizacji startowej $ - $ w przypadku utracenia połączenia z kontrolerem BSP automatycznie powróci do miejsca startowego wykorzystując odpowiednie algorytmy nawigacyjne i zapisaną lokalizację startową.
    \item Kalibracja przez użytkownika $ - $ procedura kalibracji BSP będzie intuicyjna i będzie możliwa do przeprowadzenia przez użytkownika bez żadnych kwalifikacji. Razem z dronem dostarczana będzie instrukcja kalibracji "krok po kroku".
    \item Diody kontrolne - każde ramię z silnikiem zostanie wyposażone w diodę kontrolną. Diody te będą ułatwiały proces kalibracji, a podczas lotu będą zwiększały widoczność BSP
\end{enumerate}
\section{Gdzie system jest wykorzystywany (GDZIE)}
Korzystać z systemu można w obszarach zamkniętych jak i otwartych. Między innymi: obszary zurbanizowane, terenty wiejskie, obszary leśne oraz górskie. Urządzenie nie nadaje się do korzystania w wodzie. 

\subsection{Ograniczenia systemu.}
Korzystając z urządzenia trzeba brać pod uwagę czynniki takie jak: 
\begin{itemize}
    \item Pogoda $-$ przy dużym wietrze mogą wystąpić problemy ze sterownością, przy wzmożonym deszczu może dojść do zwarć w systemie, bądź w momencie burz do uderzenia piorunem. W sytuacji dużego zachmurzenia lub mgły obraz z kamery może być niewyraźny oraz jest możliwe utrata widoczności drona. Korzystając z urządzenia w niższuch temperaturach prawdopodobne jest szybsze wyczerpanie akumulatora. 
    \item Wysokość $-$ w momencie osiągania większych wysokości dron stanowi poważniejsze zagrożenie w momencie awarii systemu. Trzeba też brać pod uwagę możliwe kolizje z innymi statkami powietrznymi (innymi dronami, samolotami, helikopterami).
    \item Zasięg $-$ dron posiada ograniczony zasięg latania spowodowany utratą sygnału z kontrolerem na dalszych odległościach.
    \item Prawne $-$ każde państwo posiada własne regulacje prawne dotyczące latania dronami oraz innymi bezzałogowymi statkami powietrznymi takie jak limit wysokości latania, brak możliwości latania w miastach bądź nad tłumami.
\end{itemize}

\section{Dla kogo system jest przeznaczony (KTO)}
\begin{itemize}
    \item Serwisant $-$ naprawa urządzenia, wymiana części, testowanie działania systemu.
    \item Użytkownik $-$ rekreacyjne/ekstremalne latanie dronem, robienie zdjęć/filmów, kalibracja oraz ładowanie urządzenia, podgląd z kamery urządzenia na telefonie za pomocą dedykowanej aplikacji.  
\end{itemize}



\section{Przypadki uzycia}
\begin{table}[h]
\centering
\begin{tabular}{|p{5cm}|p{5cm}|p{5cm}|}
\hline
\textbf{Nazwa PU:} Wyciągnięcie karty pamięci SD & \textbf{Numer PU:} 10 & \textbf{Priorytet:} niski \\
\hline
\textbf{Aktor podstawowy:} użytkownik & \multicolumn{2}{|c|}{\textbf{Typ opisu:} ogólny} \\
\hline
\multicolumn{3}{|c|}{\textbf{Udziałowcy i cele:} Użytkownik, dron w celu przekazania karty z drona do użytkownika}\\
\hline
\textbf{Wyzwalacz: } wciśniecie płytki zawierającej karte SD & \multicolumn{2}{|c|}{\textbf{Typ wyzwalacza:} zewnętrzny} \\
\hline
\multicolumn{3}{|c|}{
    \textbf{Powiązania:} brak
}\\
\hline
\multicolumn{3}{|c|}{\textbf{Zwykły przepływ zdarzeń:}
\begin{minipage}[t]{0.6\linewidth}
    \begin{enumerate}
        \item wciśnięcie płytki zawierającej karte SD
        \item odskoczenie płytki
        \item usunięcie możliwości zapisu na karte SD w tym fotografowanie
        \item wsunięcie płytki spowrotem (poprzez użytkownika)
        \item jeśli wykryto karte to przwrócenie możliwości zapisu na karte SD
        \newline
    \end{enumerate}
\end{minipage}}\\
\hline
\multicolumn{3}{|c|}{\textbf{Przepływy poboczne: brak}}\\
\hline
\multicolumn{3}{|c|}{\textbf{Przepływy alternatywne/wyjątkowe:}
\begin{minipage}[t]{0.6\linewidth}
    \begin{enumerate}
        \item wciśnięcie płytki zawierającej karte SD
        \item nie odskoczenie płytki
        \item dostęp do karty SD poprzeze rozkręcenie drona
        \item usunięcie możliwości zapisu na karte SD w tym fotografowanie
        \item skręcenie drona spowrotem (poprzez użytkownika)
        \item jeśli wykryto karte to przwrócenie możliwości zapisu na karte SD
        \newline
    \end{enumerate}
\end{minipage}}\\
\hline
\end{tabular}
\caption{Przypadki użycia dla wyciągnięcia karty pamięci SD}
\label{tab:tabela_pu}
\end{table}


\begin{table}[h]
\centering
\begin{tabular}{|p{5cm}|p{5cm}|p{5cm}|}
\hline
\textbf{Nazwa PU:} Kalibracja drona & \textbf{Numer PU:} 11 & \textbf{Priorytet:} średni \\
\hline
\textbf{Aktor podstawowy:} użytkownik & \multicolumn{2}{|c|}{\textbf{Typ opisu:} ogólny} \\
\hline
\multicolumn{3}{|c|}{\textbf{Udziałowcy i cele:} Użytkownik, dron}\\
\hline
\textbf{Wyzwalacz: Wciśnięcie przycisku służacego do kalibracji na dronie} & \multicolumn{2}{|c|}{\textbf{Typ wyzwalacza:} zewnętrzny} \\
\hline
\textbf{Powiązania:} brak\\
\hline
\multicolumn{3}{|c|}{\textbf{Zwykły przepływ zdarzeń:}
\begin{minipage}[t]{0.6\linewidth}
    \begin{enumerate}
        \item wciśnięcie przycisku rozpoczynającego kalibracje trzymając go prosto, poziomo
        \item zaświecenie się lampki kontrolnej na zielono
        \item obrócenie drona względem osi $z$ o $90\%$
        \item zaświecenie się lampki kontrolnej na zielono
        \item obrócenie drona względem osi $z$ o $90\%$
        \item zaświecenie się lampki kontrolnej na zielono
        \item obrócenie drona względem osi $z$ o $90\%$
        \item zaświecenie się lampki kontrolnej na zielono
        \item obrócenie drona względem osi $z$ o $90\%$
        \item zaświecenie się lampki kontrolnej na zielono
        \item powrót do punktu 3 tym razem względem osi $x$, kontynuacja do punktu 10, po czym powtórzenie względem osi $y$
        \item zakończenie kalibracji
        \item testowanie lotu poprzez użytkownika, jeśli efekt nie zadowalający powrót do punktu pierwszego \newline
        
    \end{enumerate}
\end{minipage}}
\\
\hline
\multicolumn{3}{|c|}{\textbf{Przepływy poboczne: brak}}\\
\hline
\multicolumn{3}{|c|}{\textbf{Przepływy alternatywne/wyjątkowe:}
\begin{minipage}[t]{0.6\linewidth}
    \begin{enumerate}
        \item wciśnięcie przycisku rozpoczynającego kalibracje        
        \item nieudana kalibracja
        \item zaświecenie się kontrolek na czerwono
        \item wyłączenie trybu kalibracji
        \newline
    \end{enumerate}
\end{minipage}}\\
\hline
\end{tabular}
\caption{Przypadki użycia dla wyciągnięcia karty pamięci SD}
\label{tab:tabela_pu}
\end{table}


\begin{table}[h]
\centering
\begin{tabular}{|p{5cm}|p{5cm}|p{5cm}|}
\hline
\textbf{Nazwa PU: }Ładowanie akumulatorów drona  & \textbf{Numer PU:} 14 & \textbf{Priorytet:} średni \\
\hline
\textbf{Aktor podstawowy:} użytkownik & \multicolumn{2}{|c|}{\textbf{Typ opisu:} ogólny} \\
\hline
\multicolumn{3}{|c|}{\textbf{Udziałowcy i cele:} Użytkownik, dron w celu naładowania akumulatorów drona}\\
\hline
\textbf{Wyzwalacz: } podpięcie kabla USB-C (podłączonego do zasilania) do drona & \multicolumn{2}{|c|}{\textbf{Typ wyzwalacza:} zewnętrzny} \\
\hline
\multicolumn{3}{|c|}{
\begin{minipage}[t]{0.6\linewidth}
    \begin{itemize}
    \item \textbf{Powiązania:} brak
    \item \textbf{Asocjacja:} brak
    \item \textbf{Zawieranie:} brak
    \item \textbf{Rozszerzenie:} brak
    \item \textbf{Generalizacja:} brak
    \end{itemize}
\end{minipage}}\\
\hline
\multicolumn{3}{|c|}{\textbf{Zwykły przepływ zdarzeń:}
\begin{minipage}[t]{0.6\linewidth}
    \begin{enumerate}
        \item podpięcie kabla USB-C (podłączonego do zasilania) do drona
        \item rozpoczęcie procesu ładowania akumulatorów
        \item osięgniecie maskymalnej pojemnośći akumulatorów
        \item wyciągneicie kabla zasilającego
        \newline
    \end{enumerate}
\end{minipage}}\\
\hline
\multicolumn{3}{|c|}{\textbf{Przepływy poboczne:}
\begin{minipage}[t]{0.6\linewidth}
    \begin{enumerate}
        \item podpięcie kabla USB-C (podłączonego do zasilania) do drona
        \item rozpoczęcie procesu ładowania akumulatorów
        \item wyciągneicie kabla zasilającego
        \newline
    \end{enumerate}
\end{minipage}}\\
\hline
\multicolumn{3}{|c|}{\textbf{Przepływy alternatywne/wyjątkowe:} brak}\\
\hline
\end{tabular}
\caption{Przypadki użycia dla ładowania drona }
\label{tab:tabela_pu}
\end{table}



\begin{table}[h]
\centering
\begin{tabular}{|p{5cm}|p{5cm}|p{5cm}|}
\hline
\textbf{Nazwa PU: }Ładowanie akumulatorów kontrolera  & \textbf{Numer PU:} 15 & \textbf{Priorytet:} średni \\
\hline
\textbf{Aktor podstawowy:} użytkownik & \multicolumn{2}{|c|}{\textbf{Typ opisu:} ogólny} \\
\hline
\multicolumn{3}{|c|}{\textbf{Udziałowcy i cele:} Użytkownik, kontroler w celu naładowania akumulatorów kontrolera}\\
\hline
\textbf{Wyzwalacz: } podpięcie kabla USB-C (podłączonego do zasilania) do kontrolera & \multicolumn{2}{|c|}{\textbf{Typ wyzwalacza:} zewnętrzny} \\
\hline
\multicolumn{3}{|c|}{
\begin{minipage}[t]{0.6\linewidth}
    \begin{itemize}
    \item \textbf{Powiązania:} brak
    \item \textbf{Asocjacja:} brak
    \item \textbf{Zawieranie:} brak
    \item \textbf{Rozszerzenie:} brak
    \item \textbf{Generalizacja:} brak
    \end{itemize}
\end{minipage}}\\
\hline
\multicolumn{3}{|c|}{\textbf{Zwykły przepływ zdarzeń:}
\begin{minipage}[t]{0.6\linewidth}
    \begin{enumerate}
        \item podpięcie kabla USB-C (podłączonego do zasilania) do kontrolera
        \item rozpoczęcie procesu ładowania akumulatorów
        \item osięgniecie maskymalnej pojemnośći akumulatorów
        \item wyciągneicie kabla zasilającego
        \newline
    \end{enumerate}
\end{minipage}}\\
\hline
\multicolumn{3}{|c|}{\textbf{Przepływy poboczne:}
\begin{minipage}[t]{0.6\linewidth}
    \begin{enumerate}
        \item podpięcie kabla USB-C (podłączonego do zasilania) do kontrolera
        \item rozpoczęcie procesu ładowania akumulatorów
        \item wyciągneicie kabla zasilającego
        \newline
    \end{enumerate}
\end{minipage}}\\
\hline
\multicolumn{3}{|c|}{\textbf{Przepływy alternatywne/wyjątkowe:} brak}\\
\hline
\end{tabular}
\caption{Przypadki użycia dla ładowania kontrolera }
\label{tab:tabela_pu}
\end{table}


\begin{table}[h]
\centering
\begin{tabular}{|p{5cm}|p{5cm}|p{5cm}|}
\hline
\textbf{Nazwa PU: }Wysyłanie obrazu z kamery na żywo  & \textbf{Numer PU:} 16 & \textbf{Priorytet:} średni \\
\hline
\textbf{Aktor podstawowy:} dron & \multicolumn{2}{|c|}{\textbf{Typ opisu:} szczegółowy} \\
\hline
\multicolumn{3}{|c|}{\textbf{Udziałowcy i cele:} dron oraz kontroler z podłaczonym telefonem w celu udostępnienia możliwości poglądu widoku z drona na żywo}\\
\hline
\textbf{Wyzwalacz: } uruchomienie drona & \multicolumn{2}{|c|}{\textbf{Typ wyzwalacza:} zewnętrzny} \\
\hline
\multicolumn{3}{|c|}{
\textbf{Powiązania:} brak}
\\
\hline
\multicolumn{3}{|c|}{\textbf{Zwykły przepływ zdarzeń:}
\begin{minipage}[t]{0.6\linewidth}
    \begin{enumerate}
        \item włączenie drona
        \item rozpoczęcie wysyłania wideo
        \item odbiór wideo poprzez drona
        \item wyświetlanie wideo na dronie
        \newline
    \end{enumerate}
\end{minipage}}\\
\hline
\multicolumn{3}{|c|}{\textbf{Przepływy poboczne:} brak}
\\
\hline
\multicolumn{3}{|c|}{\textbf{Przepływy alternatywne/wyjątkowe:} brak}\\
\hline
\end{tabular}
\caption{Przypadki użycia dla przesyłania obrazu na żywo }
\label{tab:tabela_pu}
\end{table}

\end{document}